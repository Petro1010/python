\documentclass{article}

\usepackage{xurl}
\usepackage{booktabs}
\usepackage{tabularx}
\usepackage{float}

\title{SE 3XA3: Development Plan\\Ultimate Calculator}

\author{Group 15 L01
		\\ Mathew Petronilho, petronim
		\\ Jarod Rankin, rankij5
		\\ Logan Brown, brownl33
		\\ Syed Bokhari, bokhars
}

\date{}

%\input{../Comments}

\begin{document}

\begin{table}[hp]
\caption{Revision History} \label{TblRevisionHistory}
\begin{tabularx}{\textwidth}{llX}
\toprule
\textbf{Date} & \textbf{Developer(s)} & \textbf{Change}\\
\midrule
February 3, 2022 & Mathew Petronilho & Added Introduction, Meeting Plan and Project Schedule\\
February 3, 2022 & Jarod Rankin & Added Git Workflow plan and Technology\\
February 3, 2022 & Logan Brown & Added Team Communication Plan and Proof of Concept Demonstration Plan\\
February 3, 2022 & Syed Bokhari & Added Team Member Roles and Coding Style\\
February 4, 2022 & Mathew Petronilho & Punctuation and Grammar Update \\
... & ... & ...\\
\bottomrule
\end{tabularx}
\end{table}

\newpage

\maketitle

The following document will contain the development plan for our Ultimate Calculator project, including team organization and implementation of the project.

\section{Team Meeting Plan}
\subsection{Meeting Schedule}
\begin{table}[hbt!]
\begin{tabular}{|c|c|c|}
\hline
Where           & When                & Frequency                  \\ \hline
Microsoft Teams & 9:30 am to 11:30 am & Every Tuesday and Thursday \\
Discord         & 8 pm to 10 pm       & Every Monday               \\ \hline
\end{tabular}
\caption{Team Meeting Schedule}
\label{tab:my-table}
%\vspace{-17mm}
\end{table}

\subsection{Meeting Roles}
\begin{enumerate}
    \item \underline{Note Taker:} This person's role is to record any key decisions that have been decided on, any problems that have been encountered, and keep track of results of the meetings. They must ensure that all these results are shared with the other members through well maintained documents.
    \item \underline{Time Keeper:} This person's role is to keep track of meeting dates and durations and record them into a shared document. They must also ensure the meeting schedules are followed and that time limits for specific topics are respected.
    \item \underline{Group Atmosphere Monitor:} This person's role is to watch the tone of the conversation and ensure the topic of the meetings stays project related. They will help to settle conflicts and speak up when things get out of hand.
    \item \underline{Visionary:} This person’s role is to ensure what is being done is helping the project move along and contributes to overall productivity. They will guide the group through the discussion and make sure that everyone is participating.
    \item \underline{Participants:} This person’s role is to punctually attend each meeting and to contribute to the workflow of the project. They must be respectful and communicate effectively and concisely.
    
\end{enumerate}

\subsection{Agenda Rules}
The agenda will contain topics to discuss for a particular meeting, along with the allotted times for each topic. If any activities or resources are required for a part of the meeting, they will be recorded in the agenda. Attendance will also be recorded in the agenda at the start of each meeting.

\section{Team Communication Plan}

All meetings for this project will take place over Microsoft Teams or Discord as discussed in the meeting plan. Outside of meetings, all communication will take place over a private chat on Discord. Team members are expected to check this chat at least two times a day.

\begin{table}[h]
\label{tab:my-table}
\begin{tabularx}{\textwidth}{|X|X|X|}
\hline
\textbf{Name}    & \textbf{Microsoft Teams /} & \textbf{Discord}     \\
& \textbf{Gitlab} &\\ \hline
Syed Bokhari       & bokhars@mcmaster.ca      & Fyke        \\
Mathew Petronilho & petronim@mcmaster.ca     & Matt\_Petro \\
Jarod Rankin       & rankij5@mcmaster.ca      & jarodr11    \\
Logan Brown        & brownl33@mcmaster.ca     & 1brownlog   \\ \hline
\end{tabularx}
\caption{Member Contact Information}
\end{table}

\newpage

\section{Team Member Roles}
The following table highlights the assigned roles for each member of the team:

\begin{table}[H]
    \centering
    \begin{tabularx}{\textwidth}{|X|X|}
\hline
        \textbf{Name} & \textbf{Role(s)}\\
\hline
        Syed Bokhari                    & Participant\\
                                    & Note Taker\\
                                    & Developer\\
                                    & Tester\\
			    & Git Expert\\
\hline
        Mathew Petronilho                  & Participant\\
                                    & Visionary\\
                                    & Developer\\
                                    & Tester\\
			    & Python Expert\\
\hline
        Jarod Rankin                     & Participant\\
                                    & Group Atmosphere Monitor\\
                                    & Developer\\
                                    & Tester\\
			    & Documentation Expert\\
\hline
       Logan Brown                   & Participant\\
                                    & Time Keeper\\
                                    & Developer\\
                                    & Tester\\
			    & Design Expert\\
\hline
    \end{tabularx}
\caption{Member Roles} \label{tab:memberRoles}

\end{table}


\section{Git Workflow Plan}
Our project will follow a feature-branch model where each new feature is pushed to their respective branches. These branches will be merged with the main branch only when the feature is complete and functional. The branches will be updated with the main branch daily to prevent future merge errors. Every commit made to the repo should have a descriptive message to identify what each commit has changed.
All final commits for a project milestone will be tagged in accordance with the tag provided in class.

\section{Proof of Concept Demonstration Plan}
\subsection{Significant Risks}
\subsubsection{Implementation}
The implementation of the GUI will be the most difficult part as we don’t have much experience creating our own. The backend side of the project should not be as difficult since we already have the domain knowledge for the calculations.

\subsubsection{Testing}
Testing the correctness of the calculations is not a concern since those always have a single acceptable answer. Testing of the GUI however is more difficult as there are many unpredictable ways a user can interact with a GUI which we may not catch.

\subsubsection{Library Installation}
The GUI library pyqt5 is well known and should be straightforward to install.

\subsubsection{Portability}
Portability is not a concern since python and the associated libraries being used are cross platform.


\subsection{Overcoming Risks}
\subsubsection{Implementation}
To overcome the risk of difficulty in implementing a GUI, we will use a standard library pyqt5 which is known for being simple to use and understand. This library was used in the previous implementation which we can use to learn how the library is used and how we can improve it. We will also conduct our own research on how to use pyqt5 through tutorials and online resources.

\subsubsection{Testing}
Testing of the GUI can be achieved mainly through manual testing. To ensure a large variety of cases are tested, all team members will conduct their own tests individually. In addition to this, we will conduct both functional testing (calculations, I/O, dividing by zero, etc.) and non-functional testing (responsiveness, resizing window, ease of navigation, etc.)   

\subsubsection{Library Installation}
All team members will have to install the necessary libraries for testing purposes, so if any issues arise team members can help each other out.

\subsubsection{Portability}
Team members will use the same, up to date versions of python and necessary libraries and ensure that they are all cross platform.

\section{Technology}
\subsection{Programming Language}
Our project will be written in a current version of Python. Python was chosen as all group members have familiarity with the language and Python’s various libraries can help us in implementing a better design.
\subsection{IDE}
The IDE most group members will be using is Visual Studio Code, as it is easy to use and it allows us to commit and push our code directly from the application. Group members may also use a combination of various other IDEs and text editors such as PyCharm and Sublime Text.
\subsection{Documentation}
For documentation purposes the group will be using a combination of Google Docs and Overleaf to produce \LaTeX\ documents. These documents can be found in the project repository on GitLab. Any added features to the source code will also be documented using doxygen.
\subsection{Testing}
To test our project we will use pytest due to the language of choice and the familiarity for each group member. We can also test our GUI component through various manual testing practices.


\section{Coding Style}
The code will follow the style specified by Python.org. This can be found at \url{https://www.python.org/dev/peps/pep-0008/}. The guidelines will be used to create an easily readable environment with the addition of code simplicity. The variables will be named using camel case, while the classes and methods will be named using pascal case. Comments will be used to document the code as it’s written and will be updated in accordance with code changes. The comments should be easily understandable and limited to a length of 72 characters. Complete sentences will be used starting with a capital letter.

\section{Project Schedule}

The project schedule takes the form of a Gantt Chart and can be viewed here: \url{https://gitlab.cas.mcmaster.ca/petronim/ultimate_calculator_l01_group15/-/tree/main/UltimateCalculator/ProjectSchedule/3XA3ProjectPlan.pdf}.
This schedule will be updated regularly through out the project duration as deliverables become more clear.

\section{Project Review}
Not yet applicable

\end{document}